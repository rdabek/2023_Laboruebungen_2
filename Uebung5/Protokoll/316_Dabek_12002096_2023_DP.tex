\documentclass{article}

% ----------------------------------------------------------------
% Packages
\usepackage{graphicx}
\usepackage{caption}                    % for captions
\usepackage[ngerman]{babel}             % german language support
\usepackage[T1]{fontenc}                % fix non asci letter encoding
\usepackage[nottoc,numbib]{tocbibind}   % for numbering list of figures
                                        % and images and equations
% ----------------------------------------------------------------

% ----------------------------------------------------------------
% Setup bibliography references style
\bibliographystyle{ieeetr}
% ----------------------------------------------------------------

% ----------------------------------------------------------------
% Make table of equations
% \newcommand{\listequationname}{Formelverzeichnis}
% \newlistof{equation}{eq}
% \newcommand{\equations}[1]{%
%     \refstepcounter{equation}
%     \par\noindent\textbf{Formel \theequation. #1}
%     \addcontentsline{eq}{equation}{
%         \protect\numberline{\thechapter.\theequation}#1
%     }
%     \par
% }
% ----------------------------------------------------------------

% ----------------------------------------------------------------
% Add numbering to list of figures, list of tables
\renewcommand{\listoffigures}{%
    \begingroup
    \let\section\subsection % change section to be subsection
    \tocsection
    \tocfile{\listfigurename}{lof}
    \endgroup
}
\renewcommand{\listoftables}{%
    \begingroup
    \let\section\subsection % change section to be subsection
    \tocsection
    \tocfile{\listtablename}{lot}
    \endgroup
}
% ----------------------------------------------------------------

% ----------------------------------------------------------------
% Title, author and date
\title{\textbf{Laborprotokoll Drehpendel}}
\author{
    Rafał Dąbek \\
    Lucas Hörl
}
\date{
    \today \\
    \textit{Versuchsdurchführung am 29. November 2023}
}
% ----------------------------------------------------------------

% ----------------------------------------------------------------
% Make links in the document
\usepackage{hyperref}                   % for links
% Setup links
\hypersetup{
    colorlinks,
    citecolor=black,
    filecolor=black,
    linkcolor=black,
    urlcolor=black
}
% ----------------------------------------------------------------

\begin{document}

% First page with title, author and date
\maketitle
\clearpage

% Second page with table of contents
\tableofcontents
\clearpage

\section{Einleitung}
\clearpage

\section{Versuchsaufbau}
\clearpage

\section{Theoretische Grundlagen des Pendels}
\subsection{Differentialgleichung}
% \noteworthy{a=b}{Pendelgleichung}
\cite{texbook}
\subsection{Die Lösung der homogenen Differentialgleichung}
\subsubsection{Schwingung}
\subsubsection{Gedämpfte Schwingung}
\subsubsection{Aperiodischer Grenzfall}
\subsubsection{Kriechfall}
\subsection{Die Lösung der inhomogenen Differentialgleichung}
\subsubsection{Resonanzkurven}
\subsection{Schwebung}
\clearpage

\section{Messungen}
\begin{center}
\begin{tabular}{c c c}
    \hline
    1st & 2nd & 3rd \\
    \hline
    A & B & C \\
    D & E & F \\
    \hline
\end{tabular}
\captionof{table}{\label{demo-table}Some table.}

\begin{figure}
    \caption{This is a caption for a non-existing image}
    \end{figure}
\end{center}
\clearpage

\section{Diskussion}
\clearpage

\section{Anhang}
\subsection{Verwendete Geräte}
\listoffigures
\listoftables
\bibliography{refs.bib}

\end{document}